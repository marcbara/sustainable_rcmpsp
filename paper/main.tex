%% ============================================================
%%  main.tex  —  IJCM submission
%%  "Trading Delay Penalties for Water Compliance: ..."
%%  Compile with: pdflatex main && bibtex main && pdflatex main && pdflatex main
%% ============================================================
\documentclass[12pt,a4paper]{article}

\usepackage[utf8]{inputenc}
\usepackage[T1]{fontenc}
\usepackage{times}
\usepackage[top=2.5cm,bottom=2.5cm,left=3cm,right=3cm]{geometry}
\usepackage{setspace}
\doublespacing

\usepackage{amsmath,amssymb}
\usepackage{eurosym}            % \euro symbol
\usepackage{booktabs}
\usepackage{graphicx}
\usepackage{caption}
\usepackage{subcaption}
\usepackage{multirow}
\usepackage{array}
\usepackage{float}
\usepackage{xcolor}
\usepackage{url}
\usepackage[hidelinks]{hyperref}
\usepackage[round,authoryear]{natbib}
\usepackage{algorithmicx}       % base for algpseudocode
\usepackage{algorithm}
\usepackage[noend]{algpseudocode} % noend = cleaner pseudocode
% \usepackage{lineno}      % uncomment + \linenumbers for review submission

% ── Notation macros ──────────────────────────────────────────
\newcommand{\Z}[1]{Z_{#1}}
\newcommand{\eps}{\varepsilon}
\newcommand{\Qbar}{\bar{Q}}
\newcommand{\bx}{\mathbf{x}}
\newcommand{\blam}{\boldsymbol{\lambda}}
\newcommand{\zstar}{\mathbf{z}^*}

% ── Journal style ────────────────────────────────────────────
\bibliographystyle{apalike}

%% ============================================================
\begin{document}

\begin{titlepage}
\centering
\vspace*{2cm}
{\LARGE\bfseries Trading Delay Penalties for Water Compliance:\\
Multi-Objective Scheduling of Construction Portfolios\\
under Drought\par}
\vspace{1.5cm}
{\large
Marisa Andrea Lostumbo$^{1}$,\quad
Marc Bara Iniesta$^{2}$,\quad
Laura Calvet$^{3}$
\par}
\vspace{0.8cm}
{\normalsize
$^{1}$Universitat Oberta de Catalunya (UOC)\\
$^{2}$ESADE Business School, Universitat Ramon Llull (URL)\\
$^{3}$Universitat Aut\`{o}noma de Barcelona (UAB)\\[0.4cm]
Corresponding author: Marc Bara Iniesta, \texttt{marcoantonio.bara@esade.edu}
\par}
\vspace{1cm}
{\small\textit{To be submitted to the International Journal of Construction Management}}
\vfill
{\small February 2026}
\end{titlepage}

%% ============================================================
\begin{abstract}
\noindent
Water quota violations are a growing financial liability for construction projects
in drought-prone regions, yet existing scheduling models treat water as an unlimited
resource.  This paper introduces a multi-project scheduling framework that
explicitly balances tardiness penalties against daily water-quota violations under
stochastic drought conditions, enabling project managers to quantify and manage
water-scarcity risk at portfolio level.

Daily restriction shocks are modelled by a two-state Markov chain that captures
the persistent drought spells of Mediterranean climates.  A multi-objective
evolutionary algorithm with a water-arbitrage local-search operator generates
Pareto-optimal trade-off schedules using a pre-sampled restriction sequence.
Applied to a representative three-project construction portfolio comprising 130~activities
and 12~shared renewable resources \citep{Lostumbo2025}, the framework is evaluated
under three drought-intensity scenarios.

Under mild to moderate drought --- the regime most prevalent in current
Mediterranean conditions --- the algorithm simultaneously eliminates all tardiness
penalties while reducing water violations by up to~19.4\% relative to a naive,
water-unaware baseline schedule.  The marginal cost of water-quota compliance is
directly readable from the slope of the trade-off curve, providing a concrete input
for contract negotiations and regulatory planning.  Under severe drought, a structural collapse of the trade-off
curve reveals that scheduling optimisation loses leverage: water violations become
regime-determined, and risk mitigation must shift from scheduling to technological
interventions.  This transition point is scenario-specific and was not previously
quantifiable with existing scheduling frameworks.

The results offer practical decision support for construction managers operating
under water-restriction regulations, and demonstrate that water-aware scheduling
delivers its greatest financial value precisely in the moderate-scarcity regimes
that current climate projections identify as the most likely near-term condition
for Southern European construction.

\medskip\noindent
\textbf{Keywords:} multi-project scheduling; water scarcity; drought risk;
construction sustainability; multi-objective optimisation; evolutionary algorithm;
decision support
\end{abstract}

\newpage

%% ============================================================
\section{Introduction}
\label{sec:intro}
% Running head (for journal submission, ≤60 chars):
% "Delay penalties vs. water compliance in portfolio scheduling"

The construction sector accounts for approximately 9\% of global freshwater
withdrawals \citep{UN2021}, yet project-scheduling models almost universally treat
water as an unlimited resource.  In Mediterranean climates this assumption is
increasingly untenable: prolonged drought episodes of two to four weeks are common,
and regulatory authorities in Spain, Italy and Greece regularly issue day-specific
water quotas that penalise over-consumption.  For a mid-size construction portfolio,
unplanned quota violations can generate fines of tens of thousands of euros and
trigger cascading delays when water-intensive activities must be suspended and
rescheduled.

Classical project-scheduling research has focused on the Resource-Constrained
Multi-Project Scheduling Problem~(RCMPSP), which minimises makespan or tardiness
subject to precedence and renewable-resource constraints
\citep{Artigues2026,Hartmann2022}.  A separate, smaller strand of literature has
incorporated environmental objectives — carbon footprint, water and energy consumption
— into multi-objective formulations \citep{Barrientos2016}.  However, \emph{water
quota compliance} differs fundamentally from both streams: it is a time-varying
stochastic constraint driven by an exogenous climate process, not a fixed resource
capacity or an aggregate sustainability metric.  The scheduling decision must therefore exploit
the temporal structure of drought spells: concentrating water-intensive construction
in the free windows between restriction periods.

This paper addresses that gap by making three contributions:

\begin{enumerate}
  \item \textbf{Problem formulation.}  We introduce a bi-objective RCMPSP in which
        the two objectives are (i)~tardiness cost in euros and (ii)~daily water-quota
        violations in cubic metres.  Restriction shocks follow a two-state Markov
        chain that captures drought-spell persistence; stochasticity is handled via
        Sample Average Approximation.
  \item \textbf{Algorithm.}  We adapt MOEA/D \citep{Zhang2007} to the RCMPSP
        context, using a priority-vector representation decoded by a Serial Schedule
        Generation Scheme~(SGS) and augmented with a water-arbitrage local-search
        operator that explicitly targets restriction windows.
  \item \textbf{Managerial insight.}  We show that the value of water-aware
        scheduling is scenario-dependent: it is highest under moderate drought — the
        regime most aligned with current Mediterranean projections — and degenerates
        under extreme scarcity where violations become structurally determined by the
        climate regime rather than the schedule.
\end{enumerate}

The remainder of the paper is structured as follows.
Section~\ref{sec:lit} reviews the relevant literature.
Section~\ref{sec:model} formulates the bi-objective RCMPSP.
Section~\ref{sec:algo} describes the MOEA/D framework with the water-arbitrage
operator.
Section~\ref{sec:case} presents the case study.
Section~\ref{sec:results} reports and discusses the computational results.
Section~\ref{sec:conc} concludes.

%% ============================================================
\section{Literature Review}
\label{sec:lit}

\subsection{Resource-Constrained Multi-Project Scheduling}
\label{sec:lit:rcmpsp}

The single-project Resource-Constrained Project Scheduling Problem~(RCPSP) has been
studied for more than five decades; \citet{Artigues2026} provide a comprehensive
retrospective.  Its multi-project extension~(RCMPSP) adds shared renewable resources
and inter-project dependencies, both of which arise naturally in construction
portfolios.  Early exact approaches \citep{Pritsker1969} are tractable only for
small instances; the problem is NP-hard in general \citep{Blazewicz1983}.

Population-based metaheuristics dominate the competitive literature.
\citet{Goncalves2008} demonstrated that a genetic algorithm with an SGS decoder
outperforms exact solvers on large PSPLIB instances.  For the multi-project case,
\citet{Browning2010} analysed dependencies between concurrent projects and showed
that resource contention is the primary driver of portfolio-level delay.
\citet{Bredael2023} compared nine state-of-the-art metaheuristics on RCMPSP and
found that no single algorithm dominates across all instance families, motivating
continued algorithm development.  \citet{Lostumbo2025} studied multi-objective
construction portfolio scheduling jointly minimising project tardiness and total
water consumption; while sharing the multi-project setting, that formulation
optimises a deterministic volume metric, whereas the present work models
stochastic daily quota compliance under drought --- a distinct regulatory
constraint that consumption minimisation alone cannot enforce.

\subsection{Sustainability and Water in Construction Scheduling}
\label{sec:lit:water}

Sustainability objectives in project scheduling have been incorporated since the
early 2010s, typically through carbon-emission minimisation \citep{Xu2012} or
energy-cost trade-offs.  Water scarcity as a scheduling dimension remains almost entirely unexplored.
One of the very few studies that explicitly incorporates a water-footprint
objective in construction-related environmental optimisation is
\citet{Barrientos2016}, who applied a genetic algorithm to minimise combined
water and carbon impacts in railway infrastructure projects.  Their work,
however, treats water as an aggregate volume metric rather than a time-varying
regulatory constraint, and does not address the temporal structure of drought
spells or quota compliance.

Climate projections for Southern Europe \citep{IPCC2021} indicate a 20--40\%
reduction in summer precipitation by 2050, with increased inter-annual variability
and more frequent multi-week dry spells.  This climatic signal has direct
implications for project finance: construction contracts in Spain and Italy
increasingly contain water-restriction clauses that convert daily violations into
monetary penalties.  The financial framing shifts water from an abstract
sustainability metric to a cash-flow risk item, strengthening the case for its
inclusion in the project-scheduling objective function.

\subsection{Multi-Objective Decomposition Methods}
\label{sec:lit:moead}

Multi-objective evolutionary algorithms~(MOEAs) differ primarily in how they
maintain diversity and approximate the Pareto front.  NSGA-II \citep{Deb2002}
uses non-dominated sorting and crowding distance; SPEA2 \citep{Zitzler2001} employs
an external archive; SMS-EMOA \citep{Beume2007} selects on hypervolume contribution.
MOEA/D \citep{Zhang2007} decomposes the multi-objective problem into a set of
scalar subproblems defined by weight vectors and solves them simultaneously,
exploiting neighbourhood information to propagate improvements.  \citet{Li2009}
extended MOEA/D with differential-evolution operators, achieving state-of-the-art
results on continuous benchmark functions.

In the construction domain, \citet{Yu2025} applied MOEA/D-DE to safety and
resource-efficiency objectives in single-project scheduling, demonstrating that the
decomposition framework produces well-distributed Pareto fronts with fewer function
evaluations than NSGA-II.  These properties motivate the adoption of MOEA/D in the
present work: the scalarisation-based neighbourhood structure aligns naturally with the
bi-objective trade-off structure of the problem, and the decomposition into scalar
subproblems integrates cleanly with the domain-specific water-arbitrage operator
introduced in Section~\ref{sec:algo}.  No prior work has applied MOEA/D to RCMPSP
with a water objective.

\subsection{Research Gap}
\label{sec:lit:gap}

Table~\ref{tab:lit} maps the closest related works across three dimensions: (i)
multi-project scope, (ii) use of MOEA/D, and (iii) water as a scheduling objective.
No paper covers all three simultaneously.  In particular, the combination of
persistent stochastic water quotas, a multi-project portfolio with shared resources,
and a decomposition-based MOEA constitutes a novel research space.  The present
paper fills this gap.

\begin{table}[h]
\caption{Coverage of related work across three dimensions.
\checkmark\ = addressed; $\times$ = not addressed.}
\label{tab:lit}
\centering
\begin{tabular}{lccc}
\toprule
Reference & Multi-project & MOEA/D & Water objective \\
\midrule
\citet{Bredael2023}     & \checkmark & $\times$ & $\times$ \\
\citet{Xu2012}          & \checkmark & $\times$ & $\times$ \\
\citet{Yu2025}          & $\times$   & \checkmark & $\times$ \\
\citet{Barrientos2016}  & $\times$   & $\times$ & \checkmark \\
\citet{Lostumbo2025}     & \checkmark & $\times$ & \checkmark \\
\textbf{This paper}     & \checkmark & \checkmark & \checkmark \\
\bottomrule
\end{tabular}
\end{table}

%% ============================================================
\section{Problem Formulation}
\label{sec:model}

\subsection{Portfolio Model}
\label{sec:model:portfolio}

Let $\mathcal{P} = \{1,\ldots,M\}$ be a set of $M$~construction projects sharing a
common workforce organised into $K$~renewable resource types.  Each project
$p \in \mathcal{P}$ consists of a set of activities $\mathcal{J}_p$; let
$\mathcal{J} = \bigcup_p \mathcal{J}_p$ with $|\mathcal{J}| = N$ total activities
(including dummy source/sink nodes of zero duration).  Activity $j \in \mathcal{J}$
has:
\begin{itemize}
  \item duration $d_j \geq 0$ (working days);
  \item resource requirement $r_{jk} \geq 0$ units of resource type $k$;
  \item release date $\text{rel}_j \geq 0$ (project cannot start before
        $\text{rel}_j$).
\end{itemize}

A set of finish-to-start precedence relations $\mathcal{A}$ (including lag
variants~$\mathcal{A}^+$ where $j$ must finish at least $\ell_{ij}$ days before $i$
starts) constrains the start times $s_j \geq 0$.  The renewable capacity of
resource~$k$ is $C_k$ units per day.

Each project~$p$ has a contractual deadline $D_p$ and a daily tardiness penalty
$w_p$ (\euro{}/day).  The construction phase of project~$p$ occupies a contiguous set of
activities $\mathcal{J}_p^c \subseteq \mathcal{J}_p$; during a construction day
$t$, project~$p$ consumes $\Qbar_p$~m$^3$ of water if at least one activity of
$\mathcal{J}_p^c$ is in progress.

A feasible schedule is a vector $\mathbf{s} = (s_j)_{j \in \mathcal{J}}$ satisfying:

\begin{align}
  s_j \geq \text{rel}_j                    & \quad \forall j \in \mathcal{J}
      \label{eq:rel} \\
  s_j \geq s_i + d_i + \ell_{ij}          & \quad \forall (i,j) \in \mathcal{A}^+
      \label{eq:prec} \\
  \sum_{j: s_j \leq t < s_j + d_j} r_{jk} \leq C_k
      & \quad \forall k,\, \forall t     \label{eq:res}
\end{align}

\subsection{Stochastic Water Quota}
\label{sec:model:water}

Let $T = \max_j (s_j + d_j)$ denote the portfolio horizon.  On day~$t$, the
regulatory daily water quota is
\begin{equation}
  Q_t = \Qbar \cdot (1 - \eps_t), \qquad \eps_t \in [0,1),
  \label{eq:quota}
\end{equation}
where $\Qbar$ is a nominal aggregate quota and $\eps_t$ is a daily restriction
shock.  We model $\eps_t$ using a two-state Markov chain with states
$S \in \{R, F\}$ (Restriction, Free):
\begin{itemize}
  \item In state~$R$: $\eps_t \sim \mathrm{Uniform}(\eps^-,\eps^+)$, capturing
        partial or full restriction;
  \item In state~$F$: $\eps_t = 0$ (no restriction).
\end{itemize}
Transition probabilities $p_{RR} = 1 - 1/\mu_R$ and $p_{FF} = 1 - 1/\mu_F$
parametrise the expected spell lengths $\mu_R$ and $\mu_F$ (days).  The stationary
fraction of restriction days is $\pi_R = (1-p_{FF})/[(1-p_{RR})+(1-p_{FF})]$.

The Markov chain captures the meteorological persistence of droughts: a sequence of
independent daily shocks (Bernoulli model) would give the water-arbitrage operator
no temporal structure to exploit, since all days would be statistically
equivalent.  With spell lengths of one to three weeks, the operator can meaningfully
redirect construction activity toward the free windows between restriction episodes.
Two-state Markov chains are a standard parsimonious model for precipitation
occurrence in semi-arid climates \citep{Scanlon2023} and provide the minimal
structure required to represent drought-spell persistence without requiring
sub-daily meteorological data that is rarely available at project-planning stage.

Following the Sample Average Approximation~(SAA) paradigm \citep{Kleywegt2002}, we
generate a single realisation $(\eps_0, \ldots, \eps_{H-1})$ from the Markov chain
with a fixed seed prior to running the optimiser, then treat it as deterministic
input.  The randomness thus enters through the experimental design (different seeds
yield different realisations of the same climate scenario), not inside the
algorithm.

\subsection{Bi-Objective Problem Statement}
\label{sec:model:biobj}

Given a feasible schedule~$\mathbf{s}$ and a realised restriction sequence
$(\eps_t)$, the two objective functions are:

\begin{equation}
  \Z{1}(\mathbf{s}) = \sum_{p \in \mathcal{P}} w_p \cdot \max\!\bigl(0,\;
  C_p(\mathbf{s}) - D_p\bigr)
  \label{eq:Z1}
\end{equation}
where $C_p(\mathbf{s}) = \max_{j \in \mathcal{J}_p} (s_j + d_j)$ is the completion
time of project~$p$, and

\begin{equation}
  \Z{2}(\mathbf{s}) = \sum_{t=0}^{T-1} \max\!\bigl(0,\;
  \underbrace{\sum_{p:\,t \in \mathcal{T}_p^c(\mathbf{s})} \Qbar_p}_{\text{actual consumption}}
  -\;Q_t\bigr)
  \label{eq:Z2}
\end{equation}
where $\mathcal{T}_p^c(\mathbf{s}) = \{t : \text{at least one } j \in
\mathcal{J}_p^c \text{ is in progress at time } t\}$ is the set of active
construction days for project~$p$ under schedule~$\mathbf{s}$.

$\Z{1}$ is denominated in euros; $\Z{2}$ in cubic metres.  Both are to be
minimised.  A schedule $\mathbf{s}$ \emph{Pareto-dominates} $\mathbf{s}'$ if
$\Z{1}(\mathbf{s}) \leq \Z{1}(\mathbf{s}')$ and
$\Z{2}(\mathbf{s}) \leq \Z{2}(\mathbf{s}')$, with at least one strict inequality.
The \emph{Pareto front}~$\mathcal{F}^*$ is the set of non-dominated feasible
schedules.

The formal problem is:
\begin{equation}
  \min_{\mathbf{s} \in \mathcal{S}} \bigl(\Z{1}(\mathbf{s}),\; \Z{2}(\mathbf{s})\bigr),
  \label{eq:biobj}
\end{equation}
where $\mathcal{S}$ denotes the feasible set defined by
constraints~\eqref{eq:rel}--\eqref{eq:res}.

%% ============================================================
\section{MOEA/D with Water-Arbitrage}
\label{sec:algo}

\subsection{Solution Representation and SGS Decoder}
\label{sec:algo:rep}

A solution is represented as a priority vector $\bx = (\pi_1,\ldots,\pi_N)$, a
permutation of the $N$ activity indices.  The priority vector encodes the order in
which activities compete for resources; it is decoded to a feasible schedule by the
Serial Schedule Generation Scheme~(SGS).

The SGS processes activities in the order prescribed by $\bx$.  For each activity
$j$ in that order, it assigns the earliest start time consistent with release dates
\eqref{eq:rel}, precedence constraints~\eqref{eq:prec}, and resource
capacities~\eqref{eq:res}.  The result is a non-preemptive feasible schedule.  This
representation guarantees that every individual in the population corresponds to a
feasible schedule, eliminating the need for repair operators.

A \emph{flat-list with jump strategy} \citep{Goncalves2008} is used: at each step,
the eligible activity set (all predecessors completed, release date satisfied) is
scanned from front to back; if the first eligible activity cannot be scheduled at
the current time due to resource contention, the SGS jumps to the earliest future
time at which its resource requirements can be met, rather than blocking on it.
This variant is known to reduce makespan inflation on dense networks.

\subsection{MOEA/D Framework}
\label{sec:algo:moead}

MOEA/D \citep{Zhang2007} decomposes the bi-objective problem~\eqref{eq:biobj} into
$H$ scalar subproblems, each characterised by a weight vector
$\blam^i = (\lambda_1^i, \lambda_2^i)$ with $\lambda_1^i + \lambda_2^i = 1$.
Weight vectors are uniformly distributed over the simplex.

Each subproblem uses the weighted Tchebycheff aggregation:
\begin{equation}
  g^{\mathrm{te}}(\bx \mid \blam^i, \zstar) =
  \max_{k \in \{1,2\}} \bigl\{ \lambda_k^i \,
  |\hat{Z}_k(\bx) - z_k^*| \bigr\},
  \label{eq:tcheby}
\end{equation}
where $z_k^* = \min_{\bx \in \text{pop}} Z_k(\bx)$ is the current ideal point and
$\hat{Z}_k = Z_k / z_k^{\mathrm{nad}}$ normalises objective~$k$ by the current
nadir point $z_k^{\mathrm{nad}}$. Normalisation ensures that neither objective
dominates the scalarisation due to scale differences (euros vs.\ cubic metres).

For each subproblem~$i$, a neighbourhood $\mathcal{B}(i)$ of size~$T$ comprises
the $T$~subproblems whose weight vectors are closest to $\blam^i$ in Euclidean
distance.  Parents for reproduction are drawn from $\mathcal{B}(i)$.  After
evaluation, a child replaces the incumbent of any neighbour $j \in \mathcal{B}(i)$
for which $g^{\mathrm{te}}(\text{child} \mid \blam^j, \zstar) <
g^{\mathrm{te}}(\bx^j \mid \blam^j, \zstar)$.

\subsection{Variation Operators}
\label{sec:algo:ops}

\paragraph{Order Crossover (OX).}  Two parents $\bx^a$, $\bx^b$ produce an
offspring as follows: a contiguous segment of $\bx^a$ is inherited directly; the
remaining positions are filled in the relative order they appear in $\bx^b$,
skipping activities already placed.  OX preserves relative ordering — a useful
property for SGS-decoded scheduling since the relative priority of groups of
activities is maintained across generations \citep{Davis1985}.

\paragraph{Swap Mutation.}  With probability $1/N$ per position, two randomly
selected positions in the priority vector are exchanged.  This low mutation rate
provides exploration without disrupting the priority structure inherited from OX.

\subsection{Water-Arbitrage Local Search}
\label{sec:algo:wa}

The water-arbitrage operator targets $\Z{2}$ directly by identifying construction
activities scheduled on high-restriction days and attempting to move them to
low-restriction windows.  It operates on a decoded schedule~$\mathbf{s}$ rather
than on the priority vector, then maps the improved schedule back to a modified
priority vector via a rank-based encoding.

For each application attempt:
\begin{enumerate}
  \item Identify the construction day $t^*$ with the highest quota violation:
        $t^* = \arg\max_t \max(0,\,W_t - Q_t)$, where $W_t$ is actual water
        consumption on day~$t$.
  \item Collect the set $\mathcal{A}^*$ of construction activities in progress at
        $t^*$.
  \item For each $j \in \mathcal{A}^*$, tentatively shift $j$ to the earliest
        feasible time $t' > t^*$ such that the new schedule remains feasible and
        $\Z{2}$ does not increase.  Accept the first improving move found.
  \item Re-encode the modified schedule as a priority vector by sorting activities
        by start time, breaking ties by original priority rank.
\end{enumerate}
The operator is applied $w_a$ times per offspring (hyperparameter).  It provides
domain-specific exploitation that random permutation operators cannot replicate.

\subsection{Full Algorithm}
\label{sec:algo:full}

Algorithm~\ref{alg:moead} gives the complete procedure.

\begin{algorithm}[H]
\caption{MOEA/D with Water-Arbitrage for Bi-Objective RCMPSP}
\label{alg:moead}
\begin{algorithmic}[1]
\Require Weight vectors $\blam^1,\ldots,\blam^H$; neighbourhood size $T$;
         max generations $G$; water-arbitrage attempts $w_a$;
         restriction sequence $(\eps_t)$
\State Compute neighbourhoods $\mathcal{B}(i)$ for $i=1,\ldots,H$
\State Initialise population: for each $i$, draw $\bx^i$ uniformly at random;
       evaluate $\Z{1}(\bx^i)$, $\Z{2}(\bx^i)$ via SGS
\State Initialise ideal point $z_k^* \leftarrow \min_i Z_k(\bx^i)$, $k=1,2$
\State Initialise nadir point $z_k^{\mathrm{nad}} \leftarrow \max_i Z_k(\bx^i)$
\For{$g = 1,\ldots,G$}
  \For{$i = 1,\ldots,H$}
    \State Select two parents $a,b$ uniformly from $\mathcal{B}(i)$
    \State $\mathbf{y} \leftarrow \mathrm{OX}(\bx^a, \bx^b)$
    \State $\mathbf{y} \leftarrow \mathrm{SwapMutation}(\mathbf{y})$
    \State $\mathbf{y} \leftarrow \mathrm{WaterArbitrage}(\mathbf{y},\, w_a,\, (\eps_t))$
    \State Evaluate $\Z{1}(\mathbf{y})$, $\Z{2}(\mathbf{y})$ via SGS
    \State Update $z_k^*$ and $z_k^{\mathrm{nad}}$ if improved
    \For{each $j \in \mathcal{B}(i)$}
      \If{$g^{\mathrm{te}}(\mathbf{y} \mid \blam^j, \zstar) <
          g^{\mathrm{te}}(\bx^j \mid \blam^j, \zstar)$}
        \State $\bx^j \leftarrow \mathbf{y}$
      \EndIf
    \EndFor
  \EndFor
\EndFor
\State \Return non-dominated set of $\{\bx^1,\ldots,\bx^H\}$
\end{algorithmic}
\end{algorithm}

%% ============================================================
\section{Case Study}
\label{sec:case}

\subsection{Portfolio Description}
\label{sec:case:portfolio}

The case study uses a synthetic but realistic construction portfolio first introduced
in \citet{Lostumbo2025}, where it was designed to be representative of an
RCMPSP instance in the construction sector.  The portfolio comprises three concurrent
projects managed by a single general contractor, and is available as an open dataset
on GitHub together with the Python implementation.

Table~\ref{tab:portfolio} summarises the portfolio characteristics.  The three
projects collectively involve 130~activities~(including 6 dummy
source/sink nodes), 12~renewable resource types (architectural, engineering,
management and procurement staff), and a planning horizon of approximately
1,400~working days.

\begin{table}[h]
\caption{Portfolio characteristics. $D_p$: contractual deadline; $w_p$: daily
tardiness penalty; $\Qbar_p$: nominal daily water quota during construction phase.}
\label{tab:portfolio}
\centering
\begin{tabular}{lcccccc}
\toprule
Project & Activities & Pre-const. & Construction & Post-const. & $D_p$ (days) &
$w_p$ (\euro{}/day) \\
\midrule
P1 & 57 & 12 & 42 & 3 & 329 & 500 \\
P2 & 34 & 14 & 17 & 3 & 811 & 700 \\
P3 & 39 & 12 & 24 & 3 & 1317 & 900 \\
\midrule
\textbf{Total} & \textbf{130} & & & & & \\
\bottomrule
\end{tabular}
\end{table}

The network includes 10~finish-to-start-with-lag arcs (e.g.\ plastering must dry
for 5~days before painting commences) and 3~inter-project dependency arcs
representing shared subcontractor milestones.  The 12~resource types represent
senior and junior staff in architecture, engineering, project management, quality
assurance, legal and procurement functions; total daily availability ranges from
1~to~4~FTEs per type.  Nominal daily water consumption during the construction phase
is $\Qbar_1 = 250$\,m$^3$, $\Qbar_2 = 350$\,m$^3$, and $\Qbar_3 = 300$\,m$^3$ for
projects P1, P2 and P3, respectively.

\subsection{Drought Scenarios}
\label{sec:case:scenarios}

Three drought scenarios are defined by the parameters of the two-state Markov chain
(Section~\ref{sec:model:water}).  Table~\ref{tab:scenarios} summarises the
scenario parameters.

\begin{table}[h]
\caption{Markov chain parameters for the three drought scenarios.
$\mu_R$: expected restriction-spell length (days);
$\mu_F$: expected free-spell length (days);
$\pi_R$: stationary restriction fraction;
$[\eps^-,\eps^+]$: restriction severity range;
$\mathbb{E}[\eps_t]$: expected daily shock.}
\label{tab:scenarios}
\centering
\begin{tabular}{lcccccc}
\toprule
Scenario & $\mu_R$ & $\mu_F$ & $\pi_R$ & $[\eps^-,\eps^+]$ &
$\mathbb{E}[\eps_t]$ \\
\midrule
Mild     &  7 & 28 & 0.20 & $[0.40,\;0.60]$ & 0.10 \\
Moderate & 14 & 21 & 0.40 & $[0.55,\;0.75]$ & 0.26 \\
Severe   & 21 & 14 & 0.60 & $[0.65,\;0.90]$ & 0.47 \\
\bottomrule
\end{tabular}
\end{table}

The mild scenario (20\% restriction days, 7-day spells) represents a typical
Mediterranean summer with occasional short drought episodes.  The moderate scenario
(40\% restriction days, 14-day spells) approximates the current 30-year climatological
mean for inland Iberia during summer \citep{IPCC2021}.  The severe scenario
(60\% restriction days, 21-day spells) reflects projections for the 2050 horizon
under medium-emissions pathways \citep{IPCC2021}.  Spell durations of one to three
weeks for the restriction state and two to five weeks for the free state are
consistent with the synoptic persistence of high-pressure blocking systems that
drive summer droughts in the western Mediterranean basin.

The restriction severity ranges $[\eps^-, \eps^+]$ are calibrated to reflect
observed regulatory practice in Spain and Portugal, where drought emergency
protocols typically reduce daily water allowances by 40--90\% relative to normal
quotas \citep{Scanlon2023}.  The mild range $[0.40, 0.60]$ corresponds to
\emph{pre-alert} drought status (quotas cut by 40--60\%); the moderate range
$[0.55, 0.75]$ to \emph{alert} status; and the severe range $[0.65, 0.90]$ to
\emph{emergency} status.  The uniform distribution within each range reflects
uncertainty about the exact daily allocation, which is determined administratively
and can vary day to day within the declared status level.  These parameter values
are physically motivated placeholders; project teams should substitute
site-specific regulatory data when available.

\subsection{Computational Setup}
\label{sec:case:setup}

Algorithm parameters are given in Table~\ref{tab:params}.  The restriction
sequences were generated with seed~0 for reproducibility; the population was
initialised with a fixed random seed to ensure full experiment reproducibility.
All experiments were run on a laptop with an Intel Core i7-1165G7 processor and
16\,GB RAM under Windows~10, using Python~3.11.  The SGS decoder was implemented
in pure Python and averages 6.6\,ms per evaluation on this portfolio.

\begin{table}[h]
\caption{MOEA/D hyperparameters.}
\label{tab:params}
\centering
\begin{tabular}{lc}
\toprule
Parameter & Value \\
\midrule
Population size $H$ & 100 \\
Neighbourhood size $T$ & 20 \\
Generations $G$ & 300 \\
Water-arbitrage attempts $w_a$ & 2 \\
Crossover & Order Crossover (OX) \\
Mutation & Swap, rate $1/N$ \\
Aggregation & Weighted Tchebycheff (normalised) \\
Random seed & 0 \\
\bottomrule
\end{tabular}
\end{table}

A \emph{naive baseline} schedule is constructed by assigning activity priorities in
identity order (first-in-list priority, no water awareness) and decoding through the
SGS.  This baseline represents the outcome of a conventional scheduler that
optimises only for resource feasibility and project completion, ignoring the
restriction sequence entirely.

%% ============================================================
\section{Results and Discussion}
\label{sec:results}

\subsection{Naive Baseline}
\label{sec:results:naive}

The naive baseline produces a feasible schedule with a tardiness cost of
\euro36,500 and water violations that depend on the restriction scenario
(Table~\ref{tab:baseline}).  The entire tardiness arises from project~P1 (73~days
late at \euro500/day); projects~P2 and~P3 complete well within their deadlines owing
to their longer contractual windows (811 and~1317~days, respectively).

\begin{table}[h]
\caption{Naive baseline performance by scenario.}
\label{tab:baseline}
\centering
\begin{tabular}{lccc}
\toprule
Scenario & $\Z{1}$ (\euro{}) & $\Z{2}$ (m$^3$) & P1 delay (days) \\
\midrule
Mild     & 36,500 &  44,610 & 73 \\
Moderate & 36,500 &  98,473 & 73 \\
Severe   & 36,500 & 174,982 & 73 \\
\bottomrule
\end{tabular}
\end{table}

The baseline water violations escalate sharply with scenario severity, confirming
that the baseline scheduler is completely blind to the restriction regime: it
achieves the same tardiness in all three scenarios but incurs 3.9$\times$ more water
violations in the severe scenario than in mild.

\subsection{Pareto Fronts}
\label{sec:results:pareto}

Figure~\ref{fig:pareto} shows the Pareto fronts obtained by MOEA/D for each scenario,
alongside the naive baseline point.  Table~\ref{tab:pf} summarises key statistics.

\begin{table}[h]
\caption{MOEA/D Pareto front statistics by scenario. $|PF|$: number of
non-dominated solutions; Z$_1^{\min}$, Z$_1^{\max}$: tardiness range;
Z$_2^{\min}$, Z$_2^{\max}$: water-violation range; best reduction relative to
naive baseline.}
\label{tab:pf}
\centering
\begin{tabular}{lccccccc}
\toprule
Scenario & $|PF|$ & $Z_1^{\min}$ & $Z_1^{\max}$ & $Z_2^{\min}$ & $Z_2^{\max}$ &
$\Delta Z_1$ & $\Delta Z_2$ \\
 & & (\euro{}) & (\euro{}) & (m$^3$) & (m$^3$) & (vs naive) & (vs naive) \\
\midrule
Mild     & 5 &      0 & 20,500 &  35,976 &  37,358 & $-100\%$ & $-19.4\%$ \\
Moderate & 3 &      0 & 17,500 &  85,360 &  86,972 & $-100\%$ & $-13.3\%$ \\
Severe   & 1 &      0 &      0 & 157,024 & 157,024 & $-100\%$ & $-10.3\%$ \\
\bottomrule
\end{tabular}
\end{table}

\begin{figure}[p]
\centering
\begin{subfigure}[b]{0.65\textwidth}
  \includegraphics[width=\textwidth]{../results/paper/mild/pareto_front.png}
  \caption{Mild scenario ($\pi_R = 20\%$)}
\end{subfigure}

\medskip
\begin{subfigure}[b]{0.65\textwidth}
  \includegraphics[width=\textwidth]{../results/paper/moderate/pareto_front.png}
  \caption{Moderate scenario ($\pi_R = 40\%$)}
\end{subfigure}

\medskip
\begin{subfigure}[b]{0.65\textwidth}
  \includegraphics[width=\textwidth]{../results/paper/severe/pareto_front.png}
  \caption{Severe scenario ($\pi_R = 60\%$)}
\end{subfigure}
\caption{Pareto fronts produced by MOEA/D (filled circles) and naive baseline
(cross). Axes: $Z_1$ = tardiness penalty (\euro), $Z_2$ = water violations (m$^3$).
Note the different $Z_2$ scales across panels.}
\label{fig:pareto}
\end{figure}

\paragraph{Zero tardiness is universally attainable.}
In all three scenarios, MOEA/D finds schedules with $\Z{1} = 0$, eliminating the
\euro36,500 tardiness penalty that the naive baseline incurs.  This demonstrates
that the project network admits feasible schedules that satisfy P1's 329-day
deadline; the naive priority ordering simply fails to discover them.

\paragraph{Water-violation reductions diminish with drought severity.}
The best $\Z{2}$ improvement is 19.4\% under mild, 13.3\% under moderate, and
10.3\% under severe drought.  This monotonic decay is structurally expected: as the
fraction of restriction days $\pi_R$ increases, the low-restriction windows that the
water-arbitrage operator exploits become scarcer, limiting the reallocation margin.

\paragraph{The mild Pareto front reveals a meaningful trade-off.}
With $|PF| = 5$ solutions spanning $\Z{1} \in [0, 20{,}500]$\,\euro{} and
$\Z{2} \in [35{,}976, 37{,}358]$\,m$^3$, the mild front provides a range of
contractually distinct options.  Moving from the minimum-tardiness solution~A
(zero penalties, 37,358\,m$^3$ violations) to the minimum-water solution~C
(20,500\,\euro{} penalties, 35,976\,m$^3$ violations) achieves an additional
1,382\,m$^3$ reduction in water violations at a cost of 20,500\,\euro{} — approximately
\euro14.85 per m$^3$ saved.  This price of water-compliance represents a novel,
directly actionable metric for project finance.

\subsection{Representative Solutions for the Mild Scenario}
\label{sec:results:solutions}

Table~\ref{tab:solutions_mild} provides the three representative solutions — minimum
tardiness~(A), balanced~(B) and minimum water~(C) — for the mild scenario, with a
per-project breakdown.

\begin{table}[h]
\caption{Representative Pareto solutions — mild scenario. Constr.\ days: number
of days in the active construction phase; $\eps$-sum: total water shock accumulated
by the project; Water viol.: water violation contributed by the project (m$^3$).}
\label{tab:solutions_mild}
\centering
\makebox[0pt]{%
\begin{tabular}{llccccc}
\toprule
Solution & Project & $Z_1$ contrib.\ (\euro) & Delay (days) &
Constr.\ days & $\eps$-sum & Water viol.\ (m$^3$) \\
\midrule
\multirow{3}{*}{A — min $Z_1$}
  & P1 &      0 &  0 & 133 &  3.1 &    773 \\
  & P2 &      0 &  0 & 345 & 32.5 & 11,390 \\
  & P3 &      0 &  0 & 724 & 84.0 & 25,195 \\
\midrule
\multirow{3}{*}{B — balanced}
  & P1 & 11,500 & 23 & 157 &  6.3 &  1,577 \\
  & P2 &      0 &  0 & 345 & 28.2 &  9,871 \\
  & P3 &      0 &  0 & 724 & 83.6 & 25,078 \\
\midrule
\multirow{3}{*}{C — min $Z_2$}
  & P1 & 20,500 & 41 & 157 &  5.8 &  1,460 \\
  & P2 &      0 &  0 & 345 & 33.5 & 11,708 \\
  & P3 &      0 &  0 & 724 & 76.0 & 22,808 \\
\bottomrule
\end{tabular}}
\end{table}

Solution~A eliminates tardiness by compressing P1's construction phase to 133~active
days, which minimises overlap with restriction periods at the cost of a lower
$\eps$-sum.  Solutions~B and~C allow P1 to run longer (157~active days), increasing
P1's exposure to restriction days but permitting the water-arbitrage operator to
select lower-shock windows — evidenced by the smaller $\eps$-sum in Solution~C
(5.8) versus Solution~A (3.1 with shorter exposure but denser restriction overlap).

Project~P3 accounts for the largest water violations in all solutions, owing to its
longest construction phase (724~active days) and highest nominal consumption
(300\,m$^3$/day).  This asymmetry suggests that targeted water-mitigation contracts
for~P3 (e.g.\ greywater recycling, drought-resistant finishing materials) could
reduce portfolio-level $\Z{2}$ significantly without affecting the scheduling
decision.

\subsection{Structural Collapse under Severe Drought}
\label{sec:results:severe}

The severe scenario yields a degenerate Pareto front: a single solution with
$\Z{1} = 0$ and $\Z{2} = 157{,}024$\,m$^3$.  This collapse persists at full
production-run parameters ($H = 100$, $G = 300$) and is therefore a structural
property of the problem rather than an algorithmic artefact.

The explanation lies in the restriction fraction $\pi_R = 0.60$.  With 60\% of days
under restriction, the available free windows are too narrow and too fragmented for
the water-arbitrage operator to create meaningful differences between schedules.
Almost any feasible schedule accumulates similar cumulative water violations, since
the construction phases of P2~(345~days) and P3~(724~days) are long enough to
intersect the restriction regime regardless of how they are shifted within the
feasible region.  All non-dominated solutions therefore converge to approximately
the same $\Z{2}$ value, and MOEA/D correctly reports a single point.

This finding has a direct practical interpretation: under extreme drought, scheduling
optimisation cannot materially reduce water violations.  The \emph{portfolio-level}
mitigation strategy must then shift from scheduling (temporal reallocation of
construction activity) to technological or contractual interventions (greywater
recycling, emergency water rights, rainfall-indexed insurance).  Our framework
quantifies this transition point — the drought severity at which scheduling loses
leverage — which is itself a decision-relevant output.

Figure~\ref{fig:convergence} shows the convergence curves for the mild scenario.
Both ideal-point components ($Z_1^*$, $Z_2^*$) improve rapidly in the first
50~generations and stabilise before generation~200, confirming that $G = 300$ is
sufficient for convergence.

\begin{figure}[H]
\centering
\includegraphics[width=0.7\textwidth]{../results/paper/mild/convergence.png}
\caption{Convergence curves for the mild scenario: evolution of ideal-point
components $Z_1^*$ (tardiness, left axis) and $Z_2^*$ (water violations, right
axis) over 300 generations.}
\label{fig:convergence}
\end{figure}

\subsection{Managerial Framework}
\label{sec:results:mgmt}

The Pareto front produced by MOEA/D is an input to, not a substitute for, the
project manager's decision process.  We propose the following five-step framework
for operationalising the results in practice:

\begin{enumerate}
  \item \textbf{Scenario assessment.}  At project initiation, the project team
        selects the drought scenario consistent with the climatological outlook for
        the construction window.  In the absence of meteorological forecasts,
        the moderate scenario is recommended as the climatological baseline for
        contemporary Mediterranean construction.

  \item \textbf{Pareto front generation.}  Run MOEA/D with the scenario-specific
        restriction sequence to obtain the trade-off curve between tardiness
        penalties and water violations.

  \item \textbf{Preferred solution selection.}  In cooperation with the client,
        the project manager selects a point on the Pareto front according to the
        relative financial exposure: if contractual penalties ($w_p$) are high,
        prefer solution~A (zero tardiness); if regulatory water fines are steep,
        prefer solution~C (minimum violations); solution~B offers an intermediate
        compromise.

  \item \textbf{Price of water compliance.}  Compute the marginal cost of reducing
        water violations by one cubic metre (slope of the Pareto front between
        adjacent solutions).  Compare this to the unit regulatory fine.  If the
        marginal schedule cost exceeds the unit fine, the water-optimal schedule
        is not economically justified and solution~A is preferred on financial
        grounds alone.

  \item \textbf{Re-optimisation trigger.}  If mid-project climate conditions
        deviate significantly from the planned scenario (e.g.\ an unexpected
        prolonged drought), re-run MOEA/D with an updated restriction sequence for
        the remaining horizon, using the current partial schedule as a seed for the
        initial population.
\end{enumerate}

%% ============================================================
\section{Conclusions}
\label{sec:conc}

This paper introduced a bi-objective RCMPSP that simultaneously minimises tardiness
penalties and stochastic daily water-quota violations in a construction portfolio.
The stochastic component is modelled by a two-state Markov chain that replicates the
persistent drought spells characteristic of Mediterranean climates, and is handled
via Sample Average Approximation.  A MOEA/D algorithm with a domain-specific
water-arbitrage local-search operator solves the problem and provides a Pareto front
of trade-off schedules.

Applied to a representative three-project portfolio (130~activities, 12~resources), the
framework yielded three contributions of practical relevance:

\begin{itemize}
  \item \textbf{Tardiness elimination}: MOEA/D found zero-tardiness schedules in
        all drought scenarios, recovering up to \euro36,500 in avoidable penalties
        that the naive baseline incurred — without any relaxation of contractual
        deadlines.

  \item \textbf{Quantified water savings}: Water violations were reduced by
        10.3--19.4\% relative to the unaware baseline, depending on drought
        intensity.  Under mild and moderate drought, the framework identifies a
        clear financial price of water compliance (approximately \euro14.85/m$^3$
        in the mild case), directly usable in contract negotiations.

  \item \textbf{Structural collapse as information}: The degenerate Pareto front
        under severe drought ($|PF| = 1$) is a diagnostic output: it signals
        that scheduling optimisation has exhausted its water-reduction potential
        and that technological or contractual mitigation must take over.  This
        transition point is scenario-specific and was not previously quantifiable
        with existing scheduling frameworks.
\end{itemize}

Several directions merit future investigation.  First, the framework currently uses a
single SAA realisation; a multi-scenario SAA (10--30~realisations) would provide
confidence intervals on the Pareto front at the cost of longer computation.  Second,
the addition of a resource-levelling objective would enable a tri-objective
formulation; MOEA/D scales naturally to three objectives by extending the weight
vectors to a two-dimensional simplex.  In both cases, the decision-relevant insight
lies primarily in the shape of the trade-off and the existence of regime transitions,
rather than in the precise numerical location of individual Pareto points.

\medskip\noindent
\textbf{Data and code availability:} The portfolio dataset is publicly available
on GitHub as described in \citet{Lostumbo2025}.  The MOEA/D implementation and
experiment scripts for the present paper are available at
\url{https://github.com/marcbara/sustainable_rcmpsp}.

\medskip\noindent
\textbf{Disclosure statement:} The authors declare no conflict of interest.

\medskip\noindent
\textbf{Funding:} This research received no external funding.

%% ============================================================
\bibliography{refs}

\end{document}
